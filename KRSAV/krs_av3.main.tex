\usetheme{FINKI}
\usepackage{thumbpdf}
\usepackage{wasysym}
\usepackage{ucs}
\usepackage[T2A]{fontenc}
\usepackage[utf8]{inputenc}
\usepackage{pgf,pgfarrows,pgfnodes,pgfautomata,pgfheaps,pgfshade}
\usepackage{verbatim}
\usepackage{listings}
\usepackage{fancybox}
\usepackage{tikz}
\usetikzlibrary{shapes.callouts,decorations.pathmorphing}



\pdfinfo
{
  /Title       (AV2)
  /Creator     (Tomche Delev)
  /Author      (Tomche Delev)
}


\title[АВ3]{Аудиториски вежби 3}
\subtitle{Вовед во програмскиот јазик C\\
Oператори - продолжение\\
Внес на податоци}
\author{Концепти за развој на софтвер}
\date{}
\pgfdeclareimage[width=0.6\paperwidth]{finki_logo}{finki_name}
\titlegraphic{\pgfuseimage{finki_logo}}



\begin{document}

\frame[plain]{\titlepage}

%Automatic table of contents
%\section*{}
%\begin{frame}
%  \frametitle{Содржина}
%  \tableofcontents[section=1,hidesubsections]
%\end{frame}

\AtBeginSection[]
{
  \frame<handout:0>
  {
    \frametitle{Outline}
    \tableofcontents[currentsection,hideallsubsections]
  }
}

\AtBeginSubsection[]
{
  \frame<handout:0>
  {
    \frametitle{Outline}
    \tableofcontents[sectionstyle=show/hide,subsectionstyle=show/shaded/hide]
  }
}

\newcommand<>{\highlighton}[1]{%
  \alt#2{\structure{#1}}{{#1}}
}

\newcommand{\icon}[1]{\pgfimage[height=1em]{#1}}

\lstset{language=C,captionpos=b,
tabsize=4,frame=lines,
basicstyle=\scriptsize\ttfamily,
keywordstyle=\color{blue},
commentstyle=\color{lightgray},
stringstyle=\color{violet},
breaklines=true,showstringspaces=false}

%%%%%%%%%%%%%%%%%%%%%%%%%%%%%%%%%%%%%%%%%
%%%%%%%%%% Content starts here %%%%%%%%%%
%%%%%%%%%%%%%%%%%%%%%%%%%%%%%%%%%%%%%%%%%


\begin{frame}{Потсетување од предавања}
\begin{itemize}
	\item Оператори
	\begin{itemize}
		\item Аритметички
		\item Релациски
		\item Логички
	\end{itemize}
	\item Излез на податоци
	\texttt{printf()}
	\item Влез на податоци
	\texttt{scanf()}
\end{itemize}
\end{frame}

\begin{frame}[fragile]{Внес на податоци во C}{Функцијата \texttt{scanf}}
	\begin{verbatim}
	int scanf(Контролна_низа_од_знаци, arg1, arg2, ..., argn)	
	\end{verbatim}	
	\begin{itemize}
	\item Контролната низа од знаци е всушност низа од знаци којa ја содржи потребната информација за форматирање	
	\item arg1, arg2, ..., argn се аргументите кои ги претставуваат индивидуалните податоци
	\end{itemize}	 
\end{frame}

\begin{frame}[fragile]{Употреба на \texttt{scanf}}

	\begin{exampleblock}{Пример 1}
	\begin{lstlisting}
	#include <stdio.h>
	int main() {
	    char del;
	    int delbroj;
	    float cena;
	    scanf("%c%d%f", &del, &delbroj, &cena);
	    return 0;
	}
	\end{lstlisting}
	\end{exampleblock}

\end{frame}

\begin{frame}[fragile]{Задачa 1}
Да се напише програма за пресметување и печатење на плоштината и периметарот на круг. 
Радиусот на кругот се чита од тастатура како децимален број.
\begin{exampleblock}{Решение}
	\begin{lstlisting}
#include <stdio.h>
#define PI 3.1415
int main() {
    float r;
    float P = 0, L = 0;
    printf("Vnesete go radiusot na krugot: ");
    scanf("%f", &r);
    L = 2 * r * PI;
    P = r * r * PI;
    printf("P = %f\n", P);
    printf("L = %f\n", L);
    return 0;
}
\end{lstlisting}
\end{exampleblock}
\end{frame}


\begin{frame}[fragile]{Задача 2}
Да се напише програма која од тастатура ќе прочита два цели броја и ќе ја испечати нивната сума, 
разлика, производ и остатокот при делењето.
	\begin{exampleblock}{Решение}
		\begin{lstlisting}
		#include <stdio.h>
		int main() {
		   int a, b;
		   printf("a = ");
		   scanf("%d", &a);
		   printf("b = ");
		   scanf("%d", &b);
		   printf("a + b = %d\n", a + b);
		   printf("a - b = %d\n", a - b);
		   printf("a * b = %d\n", a * b);
		   printf("a %% b = %d\n", a % b);
		   return 0;
		}
		\end{lstlisting}
	\end{exampleblock}
\end{frame}

\begin{frame}[fragile]{Задача 3}
Да се напише програма која чита големa буквa од тастатура и ја печати истaтa како малa буквa.\\
Помош: Секој знак се претставува со ASCII број.\\
Пр. \texttt{'А' = 65, 'а' = 97}
	\begin{exampleblock}{Решение}
		\begin{lstlisting}
		#include <stdio.h>
		int main() {
		   char c;
		   printf("Vnesete golema bukva: ");
		   scanf("%c", &c);
		   printf("%c malo se pishuva %c\n", c, c + ('a' - 'A'));
		   return 0;
		}
		\end{lstlisting}
	\end{exampleblock}
\end{frame}

\begin{frame}[fragile]{Задача 4}
\begin{scriptsize}
Да се напише програма која чита знак од тастатура и во зависнот од тоа дали е мала или голема буква печати 1 или 0, соодветно.\\
Помош: Користете логички и релациони оператори за тестирање на ASCII вредноста на знакот.\\
\textbf{Бонус:} Направете проверка дали знакот е број	
\end{scriptsize}
\begin{exampleblock}{Решение}
		\begin{lstlisting}
		#include <stdio.h>
		int main() {
		    char ch;
		    int rez;
		    printf("Vnesete znak: ");
		    scanf("%c", &ch);
		    rez = (ch >= 'a') && (ch <= 'z');
		    printf("%d\n", rez);
		    return 0;
		}
		\end{lstlisting}
	\end{exampleblock}
\end{frame}

\begin{frame}[fragile]{Задача 5}
\begin{scriptsize}
Да се напише програма која ќе чита два цели броеви (x, y) од тастура и на компјутерскиот екран ќе го испечати резултатот (z) од следниот израз\\
\texttt{z = x++ + --y + (x<y)}\\
Каква вредност ќе има z за x=1, y=2?
\end{scriptsize}
	\begin{exampleblock}{Решение}
		\begin{lstlisting}
		#include <stdio.h>
		int main() {
		    int x, y, z;
		    printf("Vnesete x i y, soodvetno: ");
		    scanf("%d%d", &x, &y);
		    z = x++ + --y + (x < y);
		    printf("z = %d\n",z);
		    return 0;
		}
		\end{lstlisting}
	\end{exampleblock}
\end{frame}

\begin{frame}[fragile]{Задача 6}
Нека е дадено: \texttt{r = (x < y || y < z++)}\\
	Каква вредност ќе има r за x=1, y=2, z=3?\\
	Каква вредност ќе има z?
	\begin{exampleblock}{Решение}
	\texttt{r = 1\\z = 3}
	\end{exampleblock}
Нека е дадено: \texttt{r = (x > y \&\& y < z++)}\\
	Каква вредност ќе има r за x=1, y=2, z=3?\\
	Каква вредност ќе има z?
	\begin{exampleblock}{Решение}
	\texttt{r = 0\\z = 3}
	\end{exampleblock}
\end{frame}

\begin{frame}[fragile]{Задача 7}
Нека е дадено:
\begin{lstlisting}
	int x, y;
	y = scanf("%d", &x);
\end{lstlisting}
Каква вредност ќе има y за x=5?
	\begin{exampleblock}{Решение}
	\texttt{y = 1}
	\end{exampleblock}
Нека е дадено:
\begin{lstlisting}
	int x, y, z;
	z = scanf("%d%d", &x, &y);
\end{lstlisting}
	Каква вредност ќе има z за x=5, y=6?
	\begin{exampleblock}{Решение}
	\texttt{z = 2}
	\end{exampleblock}
\end{frame}


\begin{frame}[fragile]{Задача 8}
Да се напише програма каде од тастатура ќе се внесе цена на производ, а потоа ќе ја испечати неговата цена со пресметан ДДВ.\\
Помош: ДДВ е 18\% од почетната цена
	\begin{exampleblock}{Решение}
		\begin{lstlisting}
		#include <stdio.h>
		int main() {
		    float cena;
		    printf("Vnesete ja cenata na proizvodot: ");
		    scanf("%f", &cena);
		    printf("Vkupnata cena e %.2f\n", cena * 1.18);
		    return 0;
		}
		\end{lstlisting}
	\end{exampleblock}
\end{frame}


\begin{frame}[fragile]{Задача 9}
Да се напише програма каде од тастатура ќе се внесе цена на производ, број на рати на кои се исплаќа и 
камата (каматата е број изразен во проценти од 0 до 100). 
Програмата треба да го испечати износот на ратата и вкупната сума што ќе се исплати за производот.\\
Помош: Пресметајте ја целата сума, па потоа ратата.
\end{frame}

\begin{frame}[fragile]{Задача 9}{Решение}
	\begin{exampleblock}{Решение}
		\begin{lstlisting}
		#include <stdio.h>
		int main() {
		    float cena, kamata, rata, vkupno;
		    int brRati;
		    printf("Vnesete ja cenata na proizvodot: ");
		    scanf("%f", &cena);
		    printf("Vnesete go brojot na rati: ");
		    scanf("%d", &brRati);
		    printf("Vnesete ja kamata: ");
		    scanf("%f", &kamata);
		    vkupno = cena * (1 + kamata / 100);
		    rata = vkupno / brRati;
		    printf("Edna rata ke iznesuva: %.3f\n", rata);
		    printf("Vkupnata isplatena suma ke bide %.3f\n",vkupno);
		    return 0;
		}
		\end{lstlisting}
	\end{exampleblock}
\end{frame}


\begin{frame}[fragile]{Задача 10}
Да се напише програма каде од тастатура ќе се внесе трицифрен цел број. Програмата ќе ја испечати најзначајната и најмалку значајната цифра од бројот\\
Пример: Ако се внесе следниот бројот 795, програмата ќе испечати:\\
\texttt{	Najznacajna cifra e 7, a najmalku znacajna e 5.}\\
Помош: Искористете целобројно делење и остаток од делење.
\end{frame}

\begin{frame}[fragile]{Задача 10}{Решение}
	\begin{exampleblock}{Решение}
		\begin{lstlisting}
#include <stdio.h>
int main() {
    int broj;
    printf("Vnesete tricifren broj: ");
    scanf("%d", &broj);
    printf("Najznacajna cifra e %d, a najmalku znacajna e %d\n", broj / 100, broj % 10);
    return 0;
}
		\end{lstlisting}
	\end{exampleblock}
\end{frame}


\begin{frame}[fragile]{Задача 11}
Да се напише програма каде од тастатура ќе се внесе датумот на раѓање во формат (\texttt{ddmmgggg}). Програмата на компјутерскиот екран ќе го испечати денот и месецот на раѓање.\\
Пример: Ако се внесе следниот број \texttt{18091992}, програмата ќе испечати: \texttt{18.09}\\
Помош: Искористете целобројно делење и остаток од делење.
\end{frame}

\begin{frame}[fragile]{Задача 11}{Решение}
	\begin{exampleblock}{Решение}
		\begin{lstlisting}
		#include <stdio.h>		
		int main() {
		    int datum;
		    int den,mesec;
		    printf("Vnesete datum na raganje: ");
		    scanf("%d", &datum);
		    den = datum / 1000000;
		    mesec = (datum / 10000) % 100;
		    printf("Vasata data na raganje e %02d.%02d\n", den, mesec);
		    return 0;
		}
		\end{lstlisting}
	\end{exampleblock}
	Бонус: Задачата може да се реши и со употреба на \texttt{scanf("\%2d\%2d", \&den, \&mesec)}
\end{frame}

\begin{frame}{Материјали}{}
	Предавања, аудиториски вежби, соопштенија\\
	\href{http://courses.finki.ukim.mk/}{\textbf{courses.finki.ukim.mk}}
	\vfill
	Изворен код на сите примери и задачи\\
	\href{http://bitbucket.org/tdelev/finki-krs/}{\textbf{bitbucket.org/tdelev/finki-krs}}
	\vfill
	{\Huge Прашања ?}
\end{frame}

\end{document}
