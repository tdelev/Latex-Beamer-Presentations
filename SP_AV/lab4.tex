\documentclass[12pt,a4paper]{exam}
\usepackage{amsmath}
\usepackage{amsfonts}
\usepackage{amssymb}
\usepackage[T2A]{fontenc}
\usepackage[utf8]{inputenc}
\usepackage{listings}
\usepackage{color}

\lstset{language=C,captionpos=b,
tabsize=4,frame=lines,
basicstyle=\ttfamily,
keywordstyle=\color{blue},
commentstyle=\color{lightgray},
stringstyle=\color{red},
breaklines=true,showstringspaces=false}

\begin{document}
\pagestyle{headandfoot}
\header{\textbf{ФИНКИ\\Структурирано
програмирање}}{}{\large{\textbf{Лабораториска вежба 4}}}
\headrule
\cfoot{Страна \thepage}
\begin{center}
\Large{\textbf{Функции и рекурзија}}
\end{center}
\begin{questions}

\question
Да се напише програма во која од непзнат број на броеви кои се внесуваат од
тастатура ќе ги испечати и изброи оние броеви чии што збир на цифри е
прост број. Проверката дали даден број е прост како и збирот на цифрите на
бројот да се реализираат со посебни функции.
\\На пример:\\
\texttt{2174 -> 2 + 1 + 7 + 4 = 14 не е прост број}

\question
Во нова програма да се промени претходната програма така што наместо збирот
на сите цифри на бројот ќе проверува дали збирот на одреден број (внесен од
тастатура заедно со бројот) на цифри броејќи од десно на лево е
прост број.
\\На пример:\\
\texttt{70425 3 -> 5 + 2 + 4 = 11 е прост број}

\question
Да се напише програма која ќе ја пресметува вредноста на изразот зададен со:
\[
    \prod_{i = 1}^{n}\frac{\sum_{j = 1}^{i} 2^j}{\sum_{j = 1}^{i} a_i}
\]
каде $a_i$ e $i$-тиот Фибоначиев број.  
Пресметката на $2^i$ како и соодветните Фибоначиеви броеви да
се реализираат со посебни рекурзивни функции.

\question
Да се напише програма која ќе ги испечати сите четирицифрени броеви,
што го исполнуваат следниот услов:
\[
    \overline{abcd} = a + b^2 + c^3 + d^4
\]
Проверката на условот се врши со помош на посебна функција.

\question
Да се напише програма која за внесени два броја ќе проверува дали вториот број е подброј на првиот број.
Притоа бројот Б е подброј на бројот А ако сите цифри на Б се содржат во А по соодветниот редослед и без прекини, 
на пр. 432 е подброј на 54321, но не е подброј на 543621. 
Проверката за подброј да се реализира со рекурзивна функција.



\end{questions}
\end{document}