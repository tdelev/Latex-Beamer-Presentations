%%%%%%%%%%%%%%%%%%%%%%%%%%%%%%%%%%%%%%%%%
%%%%%%%%%% Content starts here %%%%%%%%%%
%%%%%%%%%%%%%%%%%%%%%%%%%%%%%%%%%%%%%%%%%

\section{Виртуелен деструктор}

\begin{frame}[fragile]{Пример 1}{Зошто е потребен виртуелен десктруктор?}
\lstinputlisting{src/av7/ex1.cpp}
\begin{tiny}
\begin{verbatim}
Konstruiram objekt od Osnovna
Konstruiram objekt od Izvedena
Unishtuvam objekt od Osnovna
\end{verbatim}
\end{tiny}
\end{frame}

\begin{frame}[fragile]{Пример 2}{Зошто е потребен виртуелен десктруктор?}
\lstinputlisting{src/av7/ex2.cpp}
\begin{tiny}
\begin{verbatim}
Konstruiram objekt od Osnovna
Konstruiram objekt od Izvedena
Unishtuvam objekt od Izvedena
Unishtuvam objekt od Osnovna
\end{verbatim}
\end{tiny}
\end{frame}

\section{Повеќекратно наследување}

\begin{frame}{Задача 1}
Да се напише класа Teacher која наследува особини од класите Employee и Person.
\end{frame}

\begin{frame}[fragile]{Задача 1}{Решение 1/3}
\lstinputlisting[lastline=25]{src/av7/z1.cpp}
\end{frame}

\begin{frame}[fragile]{Задача 1}{Решение 2/3}
\lstinputlisting[firstline=27,lastline=44]{src/av7/z1.cpp}
\end{frame}

\begin{frame}[fragile]{Задача 1}{Решение 3/3}
\lstinputlisting[firstline=45]{src/av7/z1.cpp}
\end{frame}

\begin{frame}{Задача 2}
Да се состави класа за автомобил со млазен погон кој наследува својства од две класи, автомобил и млазен авион.
\end{frame}

\begin{frame}[fragile]{Задача 2}{Решение 1/4}
\lstinputlisting[lastline=22]{src/av7/z2.cpp}
\end{frame}

\begin{frame}[fragile]{Задача 2}{Решение 2/4}
\lstinputlisting[firstline=23,lastline=49]{src/av7/z2.cpp}
\end{frame}

\begin{frame}[fragile]{Задача 2}{Решение 3/4}
\lstinputlisting[firstline=50,lastline=72]{src/av7/z2.cpp}
\end{frame}

\begin{frame}[fragile]{Задача 2}{Решение 4/4}
\lstinputlisting[firstline=73]{src/av7/z2.cpp}
\end{frame}

\section{Колоквиумски задачи}

\begin{frame}{Задача 1}
\begin{scriptsize}
Да се дефинира класа Casovnik, за која се чуваат информации за:
\begin{itemize}
  \item час (цел број),
  \item минути (цел број),
  \item секунди (цел број),
  \item призводител на часовникот (динамички алоцирана листа од знаци).
\end{itemize}
Од оваа класа да се изведат две нови класи \texttt{DigitalenCasovnik} и \texttt{AnalogenCasovnik}.
За дигиталниот часовник дополнително се чуваат информации за стотинките и
форматот на прикажување на времето (АМ или PM). За секоја од класите да се
дефинираат конструктори со аргументи. Во рамките на изведените класи да се
дефинира функција (Vreme) која го печати времето на дигиталниот часовник во
формат: производител, час, миннути, секунди, стотинки, АМ или PM. 
За аналогниот часовник функцијата печати: производител, час, минути, секунди.

Дополнително да се преоптовари операторот $==$ кој го споредува времето кое го
мерат два часовника и враќа \texttt{true} доколку времето на едниот часовник не
отстапува за повеќе од 30 секунди во однос на времето на другиот часовник, а во
спротивно враќа \texttt{false} (без да се води сметка за запоцнување на новиот
ден). 

Да се напише и надворешна функција (\texttt{Pecati}) која прима низа од
покажувачи кон класата \texttt{Casovnik} и нивниот број, а го печати времето на сите
часовници од низата.

\end{scriptsize}
\end{frame}

\begin{frame}[fragile,shrink=10]{Задача 1}{Решение 1/4}
\lstinputlisting[firstline=16,lastline=51]{src/av7/z3.cpp}
\end{frame}
\begin{frame}[fragile]{Задача 1}{Решение 2/4}
\lstinputlisting[firstline=52,lastline=75]{src/av7/z3.cpp}
\end{frame}
\begin{frame}[fragile]{Задача 1}{Решение 3/4}
\lstinputlisting[firstline=76,lastline=99]{src/av7/z3.cpp}
\end{frame}
\begin{frame}[fragile,shrink=10]{Задача 1}{Решение 4/4}
\lstinputlisting[firstline=101]{src/av7/z3.cpp}
\end{frame}

\begin{frame}{Задача 2}
\begin{scriptsize}
Да се дефинира класа \texttt{Otpornik} во која се чува вредноста на импедансата на
отпорникот $(Z=R)$. Класата треба да има метод кој ќе ја пресметува напонот на
краевите на отпорникот кога низ него тече дадена струја $I (U=I*|Z|)$. Од
класата отпорник да се изведе класата  \texttt{Impedansa} која ќе работи со комплексни импеданси
$(Z=R+jX)$. Класите треба да овозможуваат собирање на вредностите на две
импеданси преку операторот $+$ (сериска врска на две компоненети). Дополнително
да се преоптовари и операторот $<<$.

\end{scriptsize}
\end{frame}

\begin{frame}[fragile]{Задача 2}{Решение 1/3}
\lstinputlisting[firstline=8,lastline=35]{src/av7/z4.cpp}
\end{frame}
\begin{frame}[fragile]{Задача 2}{Решение 2/3}
\lstinputlisting[firstline=37,lastline=60]{src/av7/z4.cpp}
\end{frame}
\begin{frame}[fragile,shrink=10]{Задача 2}{Решение 3/3}
\lstinputlisting[firstline=62]{src/av7/z4.cpp}
\end{frame}



