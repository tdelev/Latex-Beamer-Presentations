\documentclass[12pt,a4paper]{exam}
\usepackage{amsmath}
\usepackage{amsfonts}
\usepackage{amssymb}
\usepackage{ucs}
\usepackage[T2A]{fontenc}
\usepackage[utf8]{inputenc}
\usepackage[english,bulgarian]{babel}
\usepackage{listings}
\usepackage{color}
\definecolor{lightgrey}{rgb}{0.9,0.9,0.9}
\usepackage[usenames,dvipsnames]{xcolor}
\input{code_style_labC.tex}

\begin{document}
\pagestyle{headandfoot}
\header{\textbf{ФИНКИ\\Објектно ориентирано
програмирање}}{}{\large{\textbf{Испитни задачи}}}
\headrule
\cfoot{Страна \thepage}
\begin{center}
\Large{\textbf{Испитни задачи}}
\end{center}
\begin{questions}
\section{1 колоквиум}

\question
Да се напише структура \texttt{Koordinata} (географска координата) која се опишува со
латитуда (децимален број), лонгитуда (децимален број) и надморска височина
(децимален број) (10 поени). Да се напише функција која пресметува растојание
помеѓу две координати според следната формула:

$rastojanie = \sqrt{(k1.latituda − k 2.latituda)^2 + (k1.longituda −
k2.longituda )^2}$

каде што \texttt{k1} и \texttt{k2} се две координати (5 поени). 

Дополнително да се напише структура \texttt{Pateka} за која ќе се чуваат
нејзиното име (низа од 50 знаци), низа од координати (максимум 500) како и број
на координати (цел број) (10 поени). Да се напишат следните функции: 
\begin{itemize}
  \item функција која прима аргумент \texttt{Pateka} и враќа вкупна должина на патеката
  (вкупното растојание помеѓу сите координати) (10 поени)
  \item функција која ги печати информациите за патеката (име и вкупна должина)
  (5 поени) 
  \item функција која како аргумент прима низа од патеки и ги печати
  информациите за најдолгата патека (10 поени)
\end{itemize}

\question

Да се дефинира класа \texttt{Film}, во која се чуваат информации за:
\begin{itemize}
  \item име (низа од 100 знаци),
  \item режисер (низа од 50 знаци),
  \item рејтинзи (динамички алоцирана низа од цели броеви),
  \item број на рејтинзи (цел број)
  \item број на гледања (цел број)
\end{itemize}
(5 поени)

За оваа класа да се напише:
\begin{itemize}
  \item default конструктор
  \item конструктор со аргументи (се прима и низа со рејтинзи)
  \item copy – конструктор и деструктор (10 поени)
\end{itemize}
Потоа да се преоптоварат следните оператори:
\begin{itemize}
  \item оператор за доделување = (5 поени)
  \item оператор ++ (prefix и postfix) кој го зголемува бројот на гледања за 1
  (5 поени)
  \item оператори >, < за споредување на два филма според просечниот рејтинг (средна
вредност на сите рејтинзи)
\item оператори ==, != за споредување на два филма според нивното име (5 поени)
\item оператор за печатење << на информациите за филмот (име, режисер, просечен
рејтинг и број на гледања) (5 поени)
\end{itemize}
Да се напише функција која како аргумент прима низа од филмови и ќе ги печати
информациите за 10-те филмовите со најголем рејтинг (10 поени).

\emph{Бонус:} Да се преоптовари операторот += за додавање нов рејтинг на филмот (цел број
од 1 до 10) (+5 поени).


\end{questions}
\end{document}