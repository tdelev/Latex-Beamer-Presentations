\chapter{Имплементација и евалуација} 

Во овој дел се опишани имплементациските детали за мобилната апликација,
веб-серверот и компонентите за интеграција со социјалните сервиси. Прототипот на мобилната апликација е развиен за
оперативниот систем Android преку користење на Android SDK кој го користи
програмскиот јазик Java. Веб-серверот е развиен со користење на рамката за
развој на веб апликации Django во програмскиот јазик Python. Целокупниот развој
е направен на преносен компјутер Lenovo ThinkPad Т400 на оперативниот систем
Ubuntu 10.04, а како околина за развој е користен Eclipse IDE. Исто така
направено е тестирање и евалуацијата на мобилната апликација и целиот систем со
користење на два мобилни уреди од производителот HTC (HTC Hero, HTC Tatoo). 

\section{Имплементација на мобилната апликација}
Следејќи ја кратката спецификација и предложената архитектура и дизајн на
мобилната компонента од системот Гео Настани во имплементациската фаза се
имплементирани следните основни кориснички барања: 
\begin{itemize}
  \item Креирање на настани
  \item Прикажување настани во околината на
корисникот
  \item Прикажување настани за кои покажале интерес контактите и
пријателите на корисникот 
  \item Пребарување настани 
  \item Означување на присуство
на некој настан 
\end{itemize}
Од овие основни кориснички барања произлегуваат и други подетални
барања кои се однесуваат претежно на собирање информации за корисничкиот
контекст како локација, пријатели и податоци од други сензори. Овие барање се:

\begin{itemize}
  \item Одредување на локацијата на корисникот
  \item Поврзување со неговите контакти и
пријатели  
\end{itemize}


Слика 16. 

Разделување на основната мобилна апликација од компонентите за
собирање на контекстот на корисникот Основните барања на системот се
имплементирани преку една основна мобилната апликација, додека за дополнителните
барања кои произлегуваат од овие основни барања а се однесуваат на контекстот на
корисникот се имплементирани како посебни компоненти (апликации) кои преку
механизмите на оперативниот систем Android се интегрираат со основната
апликација. 

За собирање на контекстот на корисникот апликацијата користат две
посебни компоненти: компонентата која се користи како извор на локацијата и
компонентата која се користи како извор за социјален контекст на корисникот,
односно поврзување со неговите пријатели

 Слика 16. 
 
 Мобилната апликација
претставува клиент на REST веб-сервиси и повеќето од имплементациските детали се
однесуваат на најдобриот начин на имплементација на ваква апликација. Пред да
започнеме со деталите околу имплементацијата може да го поставиме прашањето
зошто би развивале посебна мобилна апликација како клиент на REST веб-сервисите,
кога најчесто овие сервиси имаат свои веб-базирани апликации кои може да се
извршуваат на мобилен веб прелистувач, а со тоа и на многу од современите
мобилни уреди. Како одговор на ова прашање постојат барем пет причини со кои би
го поткрепиле изборот за развивање на посебна клиент апликација: 1.  Интеграција
со платформата на мобилниот уред. Ова означува искористување на сите можности
кои ги нуди оперативниот систем на мобилниот уред и соодветните интерфејси за
програмирање. Ова вклучува пристап до сензорите, камерата, изворите на локација
и многу други сервиси на кои не може да им се пристапи од веб прелистувач. 2. 
Во случајот на Android платформата мобилната апликација може да се интегрира со
оперативниот систем на начин со кој ќе нуди сервиси и содржина на други
апликации преку механизам кој го обезбедува самиот оперативен систем. 3. 
Апликацијата може да се извршува во позадина. Операции за кои е потребно повеќе
време (како преземање на поголеми количества податоци од веб) може да се
извршуваат во позадина. Мобилните уреди ја губат поврзаноста со мрежата, така да
апликацијата може повремено во позадина да проверува кога повторно ќе биде
воспоставена врска со веб серверот и го искористи тоа за да ги обнови
податоците. 4.  Апликацијата е многу побрза од мобилната веб апликација. Во
клиент апликација може да се имплементира кеширање на податоците кои се
преземаат од веб. Исто така се врши трансфер на помало количество податоци (само
чистите податоци во некој стандарден формат како JSON или XML) за разлика од
повлекување на овие податоци во веб апликација заедно со соодветниот HTML и
JavaScript код. 5.  Мобилната апликација може да биде конзистентна со
оперативниот систем во поглед на корисничкиот интерфејс. Овозможени се и многу
можности за иновација во корисничкиот интерфејс и креирање подобро корисничко
искуство. Секако ако на корисниците им се понуди клиентска мобилна апликација
наспроти веб апликација повеќето ќе ја изберат мобилната апликација.
 
Слика 17. Едноставна имплементација на REST клиент На Слика 17 е прикажана еден
начин на имплементација на REST клиент. На прв поглед оваа имплементација
изгледа како вистинскиот и нај едноставен начин за имплементација на ваква
апликација. Само ако на проблемот на имплементација на REST пристапиме преку
детално согледување на повеќе можни имплементации може да дојдеме до подобро
решение. Во оваа имплементација која е прилично интуитивна имплементацијата на
REST клиент може да се идентификуваат пет фази. Во првата фаза активноста која е
основната градбена единка на мобилната апликација има потреба од пристап до
некој веб-сервис и го избираа соодветниот REST метод кој го исполнува барањето.
Барањето може да биде да се земат листата на настани, да се земат детали за
некој настан или било кој метод кој е достапен преку REST веб сервисите на веб
серверот. Во вториот чекор се извршува барањето до серверот, односно се
остварува повикувањето на веб сервисот преку HTTP протоколот. Ова барање
искористување на еден од методите кои ги нуди HTTP протоколот GET (за преземање
ресурси), POST (за креирање нови ресурси), PUT (за менување ресурси) и DELETE
(за бришење ресурси). Во овој чекор се случува комуникацијата со оддалечениот
сервер и се користат комуникациските можности на телефонот како Wi-Fi или
GPRS/EDGE. Третиот чекор е обработка на податоците кои се добиени како резултат
на HTTP барањето. Во овој чекор најчесто се врши трансформација на податоците од
форматот во кој се пренесуваат податоците (JSON, XML) во објектен модел со кој
се работи во самата апликација. Во следниот четврти чекор овие податоци се
складираат во некоја податочна структура која ги чува во работната меморија и од
каде може да се прикажат на корисничкиот интерфејс или да се прави било каква
друга манипулација со нив. Последниот чекор е оној во кој податоците се
зачувуваат од работната меморија во постојана меморија. Ова е процесот на т.н.
кеширање со што ако има потреба повторно да се прикажат истите овие податоци тие
може да се пристапат од локалната меморија наместо повторно да се прави барање
до веб серверот и да се повторува целиот овој процес. Оваа постапка значително
го забрзува пристапот до податоците кои се преземени претходно. Еден од
недостатоците на овој пристап е тоа што ако се прекине извршувањето на
активноста, што не е невозможно туку напротив често се случува, ресурсите ќе се
најдат во недефинирана состојба. Во имплементацијата на оперативниот систем
Android активноста како видлива компонента од извршувањето на некоја апликација
може да биде прекината во случаи кога се извршува во позадина. Во корисничките
сценарија на мобилниот телефон не е редок случај некоја активност да се постави
во позадина и да биде заменета со поприоритетни активности како одговарање на
телефонски повик или читање кратка текстуална порака. За да се избегне овој
недостаток потребно е операцијата како што е преземањето податоци од веб
серверот и нивната обработка да се извршува на начин на кој нема да може да биде
прекината од оперативниот систем во било кој момент. Во имплементацијата за
Android ова го овозможуваат сервисите. Секоја од активностите може да стартува
сервис во кој ќе имплементира функционалност за која е од значење да не биде
прекината, а може да трае подолго време.
 
Слика 18. Имплементација на REST методи со сервис На Слика 18 е дадена една
можна имплементација на REST метод со користење на сервиси. Во оваа
имплементација значајно е користењето на сервис за операцијата која се извршува
подолго време, а исто така значајно е што апликацијата постојано има информација
за состојбата во која се наоѓаат ресурсите кои се пренесуваат преку HTTP барања.
Во првиот чекор активноста врши иницијализација на помошна класа која се
справува со стартување на сервисот и повратните резултати. Во вториот чекор се
врши стартување на сервисот и поставување на соодветните повратни функции.
Процесорот е компонентата која ја врши обработката на податоците пред да бидат
испратени до серверот, а исто така е задолжен и за конзистентноста на состојбата
на ресурсите кои се разменуваат. Конкретно во четвртиот чекор се запишува
информација за состојбата на ресурсот пред тој да се побара од веб серверот или
пак се испрати на веб серверот. Оваа информација во секое време може да ни ја
даде состојбата на секој ресурс со што и корисникот може да биде соодветно
информиран за состојбата на податоците во апликацијата. Во петтиот чекор се
извршуваат соодветните HTTP барања до веб серверот. Во овој чекор се извршуваат
операциите чие време на извршување е подолго поради комуникацијата со
оддалечениот сервер. Времето на извршување во голема мера зависи од брзината на
трансфер кој може да се постигне со комуникациските можности на мобилниот уред.
Исто така во овој чекор може да се случи да се прекине целиот овој процес на
повикување на REST методите ако мобилниот уред ја изгуби врската. Откако во
седмиот чекор ќе се извести процесорот за извршувањето на HTTP барањето кон
соодветниот REST метод, во следниот осми чекор се менува состојбата на ресурсот
во локалната база на податоци. Во овој чекор сите компоненти на корисничкиот
интерфејс кои ги прикажуваат овие податоци се освежуваат со новите податоци. Во
последните неколку чекори се врши едноставно пропагирање на резултатот до
активноста иницијатор на извршувањето на REST методот. Ваквата имплементација на
клиент апликација за REST методи не е единствената и најдобра имплементација, не
една од неколкуте можни кои ги исполнува правилата за надежна и стабилна
имплементација.

\section{Имплементација на серверот} 
 
 Беб-сервисите од архитектурниот
стил REST кој беше изложен во архитектурата се интегрираат, но и претставуваат
носечки сегмент во сервисно ориентираната архитектура. Аспектите на кои треба да
се посвети внимание при нивната имплементација се нивото на независност, односно
постигнување на слаба поврзаност како, постигнување на задоволителна
скалабилност и секако сигурноста. Слабата поврзаност (loose coupling) како
својство на софтвер е директен индикатор колкаво е нивото на поврзаност меѓу
имплементациите на различни функционалности преку различни компоненти или во
случајот веб-сервиси. Во случајот на овој систем се работи и за слаба поврзаност
меѓу имплементацијата на мобилната апликација и веб-сервисите. Ова својство на
архитектурата овозможува мобилната апликација да користи повеќе паралелни
имплементации на веб-сервиси, со што можна е интеграција со постоечки
веб-сервиси преку јавно достапни интерфејси за програмирање (API) или пак
интеграција со сопствена имплементација на веб-сервисите. Во имплементацијата на
веб-сервиси кои се потенцијални опслужувачи на милиони клиенти секогаш треба да
се размислува на скалабилноста. Скалабилност претставува својство на системот
кое овозможува да тој се справува на ист начин при зголемено оптоварување,
односно неговите перформанси да не се деградираат со расење на оптоварувањето.
Скалабилност не значи дизајнирање на систем кој ќе биде димензиониран да
опслужува конкретна голема бројка на корисници, туку значи дизајнирање на систем
кој ќе успева да се скалира т.е. расте и успева да одговара на постојано
зголемување на бројот на корисниците и нивните барања. Барањата за скалабилност
најдобро се имплементираат преку избор на соодветно скалабилна архитектура но и
со добро дизајнирана инфраструктура. Одлуките за дизајнирање на скалабилна
софтверска архитектура и нејзина имплементација на одредена инфраструктура
најчесто водат кон големо значително поедноставување на системот, затоа што во
некои случаи ова е единствениот начин да се постигне бараната скалабилност. Во
конкретната имплементација на веб-сервисите за системот за организација на
настани се очекуваат сервиси чие што извршување ќе биде во многу работи зависно
од контекстуалните информации за корисникот. Следува опис на имплементацијата на
еден ваков веб-сервис и дискусија за квалитативните придобивки на корисниците.

\subsection{Имплементација на веб-сервиси со знаење за контекстот} 

Значаен аспект во
имплементацијата на веб-сервисите е фактот дека треба да се постигне извршување
на некоја акција што ќе биде зависно и насочено од контекстот на корисникот.
Односно, тоа што треба да се постигне е имплементација на веб-сервис кој ќе може
да го обработува знаењето за контекстот и да враќа резултати кои ќе бидат во
корелација со контекстот. За да се потенцира разликата во квалитет на сервисите
кои би се извршувале со знаење за контекстот и истите сервиси без никакво знаење
за контекстот на корисникот ќе направиме споредба на две можни имплементации.
Споредбата ќе ја направиме на две од основните функционалности кои треба да се
имплементираат преку веб-сервис: 1.  Пребарување настани по клучни зборови 2. 
Препорака на настани од интерес Функционалноста за пребарувањето настани
овозможува корисниците да пребаруваат релевантни настани за клучни зборови.
Втората функционалност за препорака на настани овозможува добивање листа со
настани за кои корисникот би бил заинтересиран, а која се составува со обработка
на податоците кои ги содржат претходните активности на корисникот во системот.
  

\subsection{Пребарување настани} 

Пребарувањето е функционалност без која не може да
се замисли ниту еден квалитетен сервис. Револуционерниот напредок на пребарување
на интернет ја вгради во свеста на корисниците оваа можност за брзо и едноставно
пронаоѓање на одредени информации преку процесот на пребарување. Иако постојат
многу форми и можни имплементации на пребарување, упростената претстава на
пребарувањето која им е позната на повеќето корисници е процесот во кој тие
внесуваат некакви текстуални информации во форма на клучни зборови и добиваат
како одговор резултати кои имаат соодветна релевантност за тие клучни зборови.
Според оваа дефиниција за пребарување, имплементацијата на пребарувањето во
системот за организација и промовирање на настани би било пребарување по клучни
зборови поврзани со името, описот или местото каде се случува самиот настан.
Веб-сервисот со помош на овие клучни зборови составува прашање до базата на
податоци, по што од одговорот се добива листа на резултати подредени по нивната
релевантност за дадените клучни зборови. Имплементацијата на пребарувањето со
клучни зборови овозможува добивање резултати кои се релевантни за клучните
зборови, но најчесто само мал дел од овие резултати се релевантни за корисникот.
Да претпоставиме дека корисникот пребарува по клучните зборови “Lady Gaga” име
на популарна пејачка. Поради популарноста бројот на резултати кои ќе го добие е
многу голем (пример со користење на јавниот веб сервис на Upcoming бројката е
поголема од 80) и се состои од концерти и настани на кои пеачката има настапи
насекаде во светот. Во понатамошниот чекор останува на самиот корисник да го
пронајде во листата на резултати настанот кој би бил релевантен за неговиот
контекст. Табела 4. Квалитативна разлика во пребарувањето настани со клучните
зборови Начин на пребарување    Вкупно резултати    Релевантни резултати   
Време (секунди) Со клучни зборови поврзани со името (пример Lady Gaga)  81  3  
0.92 Со клучни зборови + локација во симболичка или апсолутна форма (пример Lady
Gaga, London)   5   3   0.40

Во Табела 4 се прикажани квалитативните разлики во резултатите од пребарувањето
без и со контекстни информации (информации за локацијата на корисникот). Во
вториот начин на пребарување значително се намалува бројот на резултати од
пребарувањето со што многу се олеснува извлекувањето на релевантните резултати
кои што во случајот се повеќе од половина од вкупниот број на резултати. Исто
така времето на пребарување е намалено за половина, што во голема мера се должи
на намалувањето на податоците кои се пренесуваат преку мрежа. Друг пример би бил
доколку корисникот би пребарувал со клучните зборови „концерт Лондон“. На овој
начин тој го подигнува нивото на интеракција со системот со тоа што поставува
контекстуално збогатено прашање. Во неколкуте клучни зборови ја вметнува и
локацијата во која би сакал да пребарува некој концерт. Со вака поставеното
прашање резултатите кои би ги добил корисникот освен што би ги исполнувале
критериумите на клучните зборови туку ќе бидат и многу по релевантни за неговиот
контекст. Преку овие примери може да се заклучи дека со вметнување на
контекстуални информации во системот за пребарување, квалитетот и релевантноста
на резултатите значително се подобрува.
 
Табела 5. Пребарување на настани поврзани со театри Начин на пребарување   
Вкупно резултати    Релевантни резултати по локација    Релевантни резултати по
локација и оддалеченост Време (секунди) Театар  7073    12  0   1.44 Театар +
Њујорк 325 27  0   3.60 Театар + локација Њујорк    1080    50  3   1.35 Театар
+ локација латитуда, лонгитуда на Њујорк 1077    50  34  1.24 Театар + локација
латитуда, лонгитуда на Њујорк + растојание 1 километар    267 68  100 1.10

Во Табела 5 се прикажани резултатите од пребарување за клучен збор „театар“ во
системо за пребарување настани. Првиот начин на пребарување е само преку
клучниот збор и на овој начин се добиваат огромен број резултати од кои тешко
може да се издвојат релевантните. Во вториот начин на пребарување се вметнува и
како клучен збор и локацијата за која пребаруваме и на овој начин го намалуваме
бројот на вкупно резултати, меѓутоа сè уште има голем број резултати. Во оваа
табела во квалитативното оценување на резултатите и нивната релевантност се
оценува и атрибутот оддалеченост на настанот од местото на пребарување. Притоа
може да се забележи дека додавање на информацијата за оддалеченост од настанот
доведува до зголемување на релевантните резултати, што е малку контрадикторно од
претходниот пример, според кој би очекувале да се намали овој број. Во
последниот пример се работи со експлицитно внесување на контекстуални информации
од страна на самиот корисник при што го имаме случајот кога апликацијата сама по
себе не е со знаење за контекстот, меѓутоа пребарувањето на некој начин го
овозможува ова. Во оваа насока може да се оди чекор понапред. Затоа што се
работи за мобилна апликација, очекувано е корисникот да е во постојано менување
на својата локација и контекст во кој се наоѓа. Знаејќи ја оваа информација и
знаејќи дека клиентската апликација има знаење за контекстот на корисникот, може
да се имплементира веб-сервис во чие извршување ќе биде многу зависно од
контекстот од мобилната апликација на корисникот. Ваквиот веб-сервис би користел
информации за локација во апсолутна форма изразени преку латитуда и лонгитуда и
резултатите кои би ги давал освен што ќе бидат иницијално релевантни, ќе може да
имаат и дополнителна информација за оддалеченоста од самиот корисник. Освен
локацијата на корисникот може да го земеме и времето во кое се случува
пребарувањето, така од резултатите ги прикажеме само оние кои се временски
релевантни односно, на пример тоа да се настаните кои се случуваат во ист ден.
5.2.3   Препорака на настани Основната цел на апликациите со знаење за
контекстот, а во случајот и веб-сервиси со знаење за контекстот е да им
овозможат на корисниците едно поинакво искуство при користењето на овие
апликации. Тоа поинакво искуство пред сè се состои од квалитетни одговори на
нивните барања, персонализирани сервиси, а притоа намалена или поедноставена
интеракција на корисникот со самиот софтвер. Во случајот на пребарување на
настани тоа би било автоматско пребарување во кое корисникот без да внесе било
какви информации би добил квалитетна и релевантна листа од настани за кои реално
би бил заинтересиран. Ова пребарување уште се нарекува и се имплементира и како
препорачување. Негова имплементација се постигнува преку по сложена обработка на
податоци кои произлегуваат од контекстот на корисникот. Во овој случај овие
информации би биле локацијата на корисникот, времето, неговите пријатели,
неговиот личен вкус и секако претходната интеракција на корисникот со системот.
Сите овие информации сочинуваат еден посложен модел на контекстот на корисникот
и доколку овој модел се разработи и врз него се применат соодветните алгоритми
може да се добие како резултат нешто што со голема веројатност е во рамките на
очекувањата на корисникот од квалитетен сервис. Можни се два пристапи во
имплементација на системот за препорачување. Првиот пристап би бил преку градење
на богат кориснички профил, во кој корисникот би внесувал повеќе информации за
својот личен вкус, интереси поврзани со разни области како филм, музика, спорт,
култура и слично. Потоа врз основа на овој профил системот може да пребарува
настани кои се поклопуваат со вкусовите на корисникот. Вториот пристап е преку
користење на техниките на колаборативно филтрирање во кои се бара поголема група
луѓе чии што вкусови се поклопуваат со вкусовите на корисникот, по што следува
препорака на некои од настаните кои за оваа група се интересни, а за кој
корисникот нема покажано интерес. Постојат и други начини со кои може да се
постигне до поквалитетни и подобро персонализирани сервиси, но многу значајно е
тоа што сите овие техники на некој начин вклучуваат информации за контекстот на
корисникот, или пак информации кои секундарно произлегуваат од контекстот на
корисникот.
