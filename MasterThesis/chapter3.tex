\chapter{Мобилни социјални мрежи и сервиси}    

Мобилните социјални мрежи овозможуваат корисниците да бидат постојано поврзани
еден со друг, да споделуваат информации и со помош на мобилната технологија да
создаваат нови мобилни заедници. Со појавувањето на iPhone во 2007, земајќи ги
предвид хардверските и софтверските можности на овој уред се овозможи сонот на
многу корисници за вистински мобилни компјутери. Во овој дел се разгледуваат
мобилните социјални мрежи од нивните првични примери се до денешните современи
сервиси и се идентификуваат и категоризираат според соодветниот вид на податоци,
начин на работа и правила на приватност.

Со преку три милијарди корисници на мобилни телефони мобилната комуникација е
незаменлив дел во секојдневниот живот на луѓето во поголем дел од светот.
Широката распространетост на оваа технологија овозможи луѓето да се чувствуваат
постојано поврзани и достапни во секое време и на секоја локација. Во многу
истражувања се разгледувани ефектите од секојдневното користење на мобилни
телефони [41][42][43][44][45].

Постојат и мислења и истражување дека мобилните телефони може да доведат до
атомизација и оддалечување меѓу корисниците со тоа што ја намалуваат
комуникацијата лице во лице во урбани околини [46][47][48].

Како што напредува и самата мобилна технологија, се развиваат нови сервиси за
мобилни телефони кои овозможуваат корисниците да создаваат, развиваат и да ги
зајакнуваат социјални врски. Слично на сајтовите на социјалните мрежи на
интернет [49][50][51][52][53], мобилните сервиси може да им помогнат на
корисниците да изградат значајни мрежи за споделување информации и други ресурси
[54].

Еден од првите мобилни социјални уреди е Lovegety во Јапонија [55]. Lovegety е
самостоен уред кој може да се носи во дланка и испушта звук кога е во близина од
околу пет метри до некој друг сличен уред со исти карактеристики. Постојат уреди
како „розова девојка“ и „сино момче“ со три можности за прилагодување: ``let's
chat'', ``let's karaoke'' или ``get2''. Кога два вакви уреди се во близина еден
до друг и се слични прилагодувања  (на пример двата уреди се поставени на
``let's chat'') испуштаат звук и светат зелено. Ако се различно прилагодени
уредите звучно ќе сигнализираат и ќе светнат црвено. Според желбите на
корисникот, ако не сака да се поврзе и комуницира со друг корисник на Lovegety
може да се исклучи звукот, да се сокрие уредот или да се преправа дека не
поседува таков уред.

Постојат два вида информации кои се споделуваат со користење на Lovegety. Прво,
информација која се однесува на идентификација на луѓето кои се заинтересирани
да го користат мобилниот уред за да среќаваат други луѓе, второ, информации што
се однесуваат на тоа каков вид на социјална интеракција бара секоја од
личностите. Информациите кои се споделуваат се прилично едноставни, но
овозможуваат луѓето да имаат интеракција со непознати во јавни простори со помош
на посредување со мобилни уреди без да откриваат лични информации како број на
мобилен телефон или пак име.

MIT Media Lab и Intel Corporation развиле две од првите мобилни социјални мрежи.
Social Serpendity е Bluetooth базиран социјален сервис на MIT развиен со цел да
ја искористи моќта на мобилната технологија и социјалните информации [56].
Social Serpendity овозможува социјална интеракција меѓу корисници на географски
блиски локации преку пронаоѓање слични кориснички профили и споделување на
информациите од овие кориснички профили. Jabberworcky е сервисот на Intel кој се
обидува да го надгледува и известува за движењето на корисниците со цел да
идентификува т.н. „познати странци“ со што би се постигнало поголемо чувство за
урбана заедница [57]. Две од овие технологии се потпираат на мобилноста на
уредите да ги одредат информациите за локацијата со цел да постигнат социјални
врски меѓу корисниците.

Раните верзии на мобилните социјални сервиси, како што се Lovegety, Seredipity и
Jabberwocky, се најчесто развивани како самостојни апликации за мобилните уреди.
Но со напредокот на мобилните уреди, постои придвижување од одделни мобилни
уреди кои ја користат социјалната поврзаност кон мобилни социјални сервиси кои
работат на мобилни телефони. Мобилните телефони станаа дел од нештата без кои
луѓето не ги напуштаат своите домови како што се клучеви и паричници. Така,
повеќето од јавно достапните мобилни социјални сервиси се развиени со можност да
се користат од мобилните телефони. Во овој дел е направен преглед на мобилни
социјални сервиси, од првите примери се до денешните и идентификувани се и
категоризирани разни мобилни социјални мрежи и сервиси. 

\section{Мобилни социјални сервиси} 

Под мобилни социјални сервиси се подразбира софтвер, апликации или системи за
мобилни телефони кои овозможуваат корисниците да се поврзуваат со други луѓе,
споделуваат информации и создаваат нови технолошки напредни мобилни заедници. Се
користат многу различни термини за да се опишат овие видови на сервиси
вклучувајќи ги и следниве: мобилна социјална мрежа, мобилен социјален софтвер,
сервис на мобилна социјална мрежа или мобилен блог.

Терминот мобилна социјална мрежа еволуира од појавувањето и издигнувањето на
социјалните мрежни сервиси (SNS) како што се Facebook, MySpace и Friendster. Во
[51] се дефинираат сајтовите на социјалните мрежи како „веб-базирани сервиси кои
им овозможуваат на индивидуалците да (1) создадат јавен или делумно јавен профил
во рамките на затворен систем, (2) да создадат листа од други корисници со кои
тие споделуваат некоја врска и (3) да ја гледаат и изминуваат нивната листа на
врски и врските направени од други во самиот систем“. Нај-очигледната разлика
меѓу мобилните социјални мрежи и стандардните социјални мрежи е во тоа што овие
вторите се веб-базирани, а додека првите се примарно наменети за мобилни
телефони и мобилни корисници.

Со дефиницијата од [51] исто така се диференцирани мобилните социјални мрежи од
мобилниот социјален софтвер. Мобилните социјални мрежи овозможуваат на
корисниците да ги идентификуваат корисниците со кои се поврзани и да ги
прелистаат нивните врски во самиот систем. Понекогаш оваа навигација се прави
преку апликација на мобилниот телефон, а понекогаш на соодветниот веб-сајт на
кој се пристапува преку компјутер. Како што напредува самата мобилна
технологија, можноста да им се пристапува на овие информации преку мобилните
уреди се зголемува, со што голем број од овие сервиси ќе бидат целосно базирани
за мобилни телефони. Мобилниот социјален софтвер се разликува од мобилните
социјални мрежи во однос на тоа што тој вообичаено се инсталира на еден мобилен
уред (или е инсталиран уште пред да се купи уредот), додека мобилната социјална
мрежа не побарува специјална апликација или програма да се извршува на телефонот
со цел да работи. Понекогаш мобилните социјални мрежи се целосно базирани на
кратки текст пораки (СМС), сервисот за мултимедијални пораки (ММС) или пренос на
говор. Мобилниот социјален софтвер не е неопходно да ги исполнува овие услови
кои се идентификуваат во дефиницијата за социјални мрежи. Овој софтвер може да
нема артикулирана листа на социјални врски меѓу корисниците, туку неговата цел е
да овозможува некаков вид социјална интеракција меѓу веќе поврзаните мобилни
корисници. На пример, и Jabberwocky и Social Serendipity се класифицираат како
мобилен социјален софтвер, а не мобилни социјални мрежи, со тоа што и двете
овозможуваат социјална интеракција меѓу групи на мобилни корисници, но не
содржат и прикажуваат разни социјални врски меѓу корисниците.

Една од другите разлики кои може да се воочат е дека мобилниот социјален софтвер
овозможува помали или поголеми групи на мобилни корисници да споделат информации
меѓу себе во динамична околина при што тие може да се на иста или различни
локациии. Притоа овие корисници најчесто немаат претходни социјални врски при
што ваквото поврзување може да им помогне да се групираат за постигнување на
некоја заедничка цел. За разлика од ова, во социјалните мрежи како на пример
Facebook, повеќето од врските се базираат на претходни познанства и
пријателства. Слично и мобилните социјални мрежи пред сè овозможуваат поврзување
со луѓето со кои веќе се познаваме или имаме некои врски во надворешниот свет.
Притоа и самите маркетинг кампањи на повеќето мобилни социјални мрежи го
поттикнуваат поврзувањето на интернет со нашите постоечки пријатели без притоа
да форсираат создавања на нови пријателства и врски.

\section{Категоризација на мобилните социјални мрежи} 

Постојат огромен број нови мобилни социјални сервиси и сите од нив имаат некои
заеднички карактеристики но и разлики по кои може да се категоризираат. Со помош
на традиционалната топологија според медиумот за пренос [58], три фактори може
да се искористат да се категоризираат овие мобилни системи, а тоа се: медиумот
за пренос, начинот на комуникација и нивото на приватност.

Првиот фактор за категоризацијата на мобилните социјални мрежи е медиумот на кој
овие системи работат. Терминот медиум означува начин на комуникација, како што
се комуникација преку мобилен телефон, компјутер или безжичен уред. И покрај
сите трендови за конвергенција на технологиите, некои мобилни социјални мрежи
единствено се базираат на мобилните телефони како начин на поврзување меѓу
корисниците. Некои други безжични уреди користат Bluetooth за поврзување, а
повеќето мобилни социјални мрежи користат некаков вид на интернет базирани
врски. Често постои и некаква веб компонента на мобилната социјална мрежа со што
таа не е целосно базирана на мобилните телефони туку се интегрира и со веб
сервиси.

Со идентификување на одреден медиум кој се користи во различни мобилни социјални
мрежи, може да се направи поделба на самите системи. Исто така значајно да се
напомене дека медиумот во мобилните социјални мрежи го налага и пренесувањето на
пораки од еден на еден (меѓусебно) до еден на многу (повеќекратен пренос).
Мобилниот телефон најчесто укажува на  начин на комуникација меѓу двајца луѓе
преку текстуални пораки или говор. Можноста мобилната социјална мрежа да
пренесува информации на повеќе луѓе претставува значајно поместување во начините
на мобилната комуникација. Можноста за пренос на пораки на повеќе луѓе одеднаш е
важен фактор во мобилните социјални мрежи затоа што овозможува група на
корисници да комуницираат брзо и едноставно.

Вториот фактор по кој се категоризираат мобилните социјални мрежи е начинот на
комуникација. Начинот на комуникација се однесува на различните форми на
комуникација на кои членовите имаат можност да вршат интеракција едни со други,
вклучувајќи текст, слики, говор (или аудио) и видео. Мобилните социјални мрежи
се разликуваат меѓу себе според начините на комуникација кои ги овозможуваат, со
тоа што некои се ограничени на неколку, за разлика од други кои овозможуваат
многу повеќе начини. Начините на комуникација го обликуваат видот на интеракција
на корисниците во мобилните социјални мрежи. На пример, понекогаш постои
ограничување на бројот на знаци кои корисниците може да го испратат преку
мобилните социјални мрежи.Секој од овие модели го одредува обликот на
комуникација кој се случува во рамките на мобилната социјална мрежа.

Третиот фактор според кој се разликуваат мобилните социјални мрежи е нивото на
приватност. Мобилните социјални мрежи овозможуваат различни нивоа на приватност
според кои комуницираат нивните корисници со нивните социјални врски. Некои
системи ги прикажуваат одредени врски на корисниците, додека други мобилни
социјални мрежи прикажуваат профили на корисниците но не и врските со други. Ако
не се прикаже врска, системот ќе се класифицира како мобилен социјален сервис
или софтвер над мобилната социјална мрежа. Друг аспект од приватноста според кои
се разликуваат е дали системите овозможуваат заемни поврзувања или овозможуваат
еднонасочни врски. На пример, некои мобилни социјални мрежи овозможуваат
корисниците да испраќаат пораки до одредени луѓе, но да примаат од други луѓе.
Некои сервиси го нарекуваат ова гледање или следење. Корисник може да следи мала
група на луѓе, но може да биде следен од многу поголема група и обратно. На
пример, популарниот НБА кошаркар Шекил Онил, на социјалната мрежа Twitter го
следат 3,164,034 луѓе, а тој следи 614 луѓе. Систем со заемни врски не
овозможува пораки во еден правец. Во овие системи, корисниците испраќаат и
примаат пораки кон и од исти луѓе. Овие специфики на мобилната социјална мрежа
го диференцираат множеството од различни сервиси кои им помагаат на корисниците
да се поврзат со други корисници на основа на слични интереси или географска
блискост.

\begin{table}[!hbp]
\caption{Категоризација на мобилни социјални сервиси}
{\scriptsize{
\begin{tabular}{|p{.1\textwidth} | p{.2\textwidth} | p{.2\textwidth} |
p{.2\textwidth} | p{.2\textwidth}|}
\hline
 Име & Опис & Медиум & Начин на комуникација & Приватност\\
 \hline
 MySay & Зборувај му на вебот & Мобилен телефон, веб & Аудио (говор), текст, 
 слики, видео & Без јавно прикажување на социјалните врски\\
 \hline
  Cromple & Одржувајте ги информирани вашите пријатели со овој едноставен блог
  систем & Веб, мобилен телефон & Текст & Еднонасочни мрежи на пријатели\\
  \hline
  Jaiku & Вашиот разговор & Веб, мобилен телефон, IM & Текст, слики &
  Еднонасочни мрежи на пријатели\\
  \hline
  Pownce & Испраќај нешто на твоите пријатели & Мобилен, веб, IM & Текст, слики,
  видео, аудио & Еднонасочни мрежи на пријатели\\
  \hline
  Twitter & Што правиш? & Мобилен, веб, IM & Текст, слики & Еднонасочни мрежи на
  пријатели\\
  \hline
  Twitxr & Сликата вреди илјада зборови & Мобилен, веб & Слики, текст &
  Еднонасочни мрежи на пријатели\\
  \hline
  Utterz & Споделете ги вашите вести & Мобилен, веб & Слика, видео, аудио,
  текст & Еднонасочни мрежи на пријатели\\
  \hline
  Dodgeball & Поврзи се со твоите пријатели & Мобилен телефон, веб & Текст &
  Мрежи со взаемни пријателства\\
  \hline
  Facebook Mobile & Користи го Facebook во движење & Примарно мобилен телефон
  (додаток на Facebook.com) & Текст, слики & Мрежи со взаемни пријателства\\
  \hline
  Loopt & Претвори го твојот мобилен во социјален компас & Мобилен телефон
  (GPS), веб, IM & Текст, слика & Мрежи со взаемни пријателства\\
  \hline
  MySpace Mobile & Место за пријатели & Примарно мобилен телефон (додаток на
  MySpace.com) & Текст & Мрежи со взаемни пријателства\\
  \hline
  Radar & Разговори со слики со твоите омилени луѓе и никој друг & Мобилен, веб,
  IM & Слики, видео, текст, аудио & Мрежи со взаемни пријателства\\
  \hline
  Socialight & Откривај информации за местата околу тебе овде и сега & Мобилен
  телефон (GPS), веб & Слики, текст & Мрежи со взаемни пријателства\\
  \hline
\end{tabular}
}}
\label{tab:categorization_mobile_services}
\end{table}

Табела 2 ги категоризира следниве мобилни социјални сервиси според видот на
медиумот, начинот на комуникација и приватност: Powence.com, Twitxr, Twitter,
Jaiku, MySay, Utterz, Cromple, Facebook Mobile, MySpace Mobile, Dodgeball,
Socialight, Loopt, Kyte и Radar. Топологијата од [58], иако од 1973 се уште е од
голема помош во категоризацијата на овие сервиси. На пример, постојат значајни
разлики во начинот на комуникација. Првите мобилни социјални мрежи, како што е
Dodgeball, се примарно текст-базирани, а денешните сервиси постојано растат и
нудат повеќе модална комуникација. На корисниците им се нуди можност да
испраќаат текст, слики, аудио и видео. На пример, MySay и Utterz се обидуваат да
се диференцираат од другите мобилни социјални сервиси со тоа што дозволуваат
корисниците да го снимаат својот говор на нивните мобилни телефони и потоа да го
објават на веб.

Оваа категоризација на мрежите исто така ја прикажува и конвергенцијата на
медиумите за комуникација. Дизајнерите на мобилните социјални мрежи развиваат
специјални апликациски верзии на нивните сервиси, кои може да се интегрираат во
други веб сајтови како што се блогови или сајтови на социјални мрежи. На пример,
луѓето не мора веќе да одат на веб сајтот на Twitter за да се приклучат, туку
може директно да се поврзат од некој друг сајт на социјална мрежа како на пример
Facebook. Не само што овие сервиси овозможуваат повеќе начини на комуникација,
туку тие исто така овозможуваат и пренос на поголем број на различни видови
податоци. Покрај севкупната конвергенција која се случува во однос на медиумите
за пренос и видовите на податоци кои се пренесуваат, сепак сè уште најголеми
мобилни социјални сервиси се оние кои произлегуваат од популарните социјални
мрежи на веб. И покрај тоа, овие сервиси се уште не се развиени како вистински
мобилен софтвер и не ја искористуваат мобилноста и контекстот на корисниците,
туку се само додаток на нивните веб сајтови наменети за мобилни уреди.

Може да се постави и прашањето дали терминот мобилни социјални мрежи е
вистинскиот начин да се опишат ваквите социјални мрежни сервиси кои овозможуваат
пристап до различни видови на податоци од социјалната мрежа на повеќе различни
технолошки платформи? И покрај конвергенцијата на сите видови на податоци,
мобилната социјална мрежа е термин кој се уште е од помош бидејќи ја потенцира
мобилноста, индивидуалноста и пристапот на системи кои се наменети за мобилни
уреди. Мобилните телефони се дефинирани преку нивната мобилност. Тие не се
врзани за кујната, канцеларијата или телефонската говорница, туку може да се
најдат на секакви различни простори. За разлика од фиксните линии, мобилните
телефони примарно се врзани за одредена личност. Луѓето најчесто поседуваат
нивни лични мобилни телефони, додека приватните фиксни линии се вообичаено
врзани за одредено место. Пристапноста, индивидуалноста и мобилноста на
мобилните телефони е она што ги прави посебно моќни комуникациски технологии и
ги прави различни од традиционалните компјутерски технологии.
