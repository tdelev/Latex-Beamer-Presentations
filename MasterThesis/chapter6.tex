\chapter{Заклучок} 

Компјутер за секој човек, компјутер на секоја маса, компјутер за секое дете,
компјутер кој е личен, персонален. Ова се некои од идеите кои пред дваесетина
години можеби звучеа не реално или научна фантастика, или пак едноставно не се
вклопувале со тогашните мислења за развојот на оваа област. Денес тоа се целосно
остварени идеи и никој не се сомнева во нивното значење и влијание во животот на
луѓето. Иако историски гледано развојот на компјутерите се менувал преку повеќе
различни компјутерски платформи, секогаш софтверските платформи и апликации биле
тие што ја издвојувале една платформа во однос на останатите. Одредени апликации
со својата масовност и со овозможувањето на значително квалитативно подобрување
во одреден сегмент од животот на луѓето се втемелуваат во свеста на корисниците
при нивното секојдневно користење на компјутерите. Не се ретки случаите кога
луѓето го поистоветуваат целокупното користење на компјутерите со некоја од овие
апликации. Впрочем голем дел од нив не можат да направат разлика помеѓу
интернетот и веб прелистувач.

Денешната ера
е дефинитивно ера во која луѓето поседуваат повеќе компјутери и се во
секојдневна интеракција со нив. Тие секојдневно се во потрага по апликации кои
ќе придонесат за подобрување во квалитетот на животот од сите аспекти. Најчесто
овие апликации се користат за подобрување на комуникацијата, зголемување на
продуктивност, но во одреден сегмент претставуваат и голема забава за луѓето.
Денешните паметни телефони се најзначајниот елемент кој ја дефинира оваа трета
ера на компјутерски платформи. Тоа не значи дека овие мали компјутери ќе ги
заменат целосно денешните персонални компјутери, тука дека тие ќе претставуваат
само дел од сите компјутери со кој луѓето се во секојдневна интеракција. Овие
компјутери во иднина ќе се среќаваат во различни форми и облици. Може да бидат
вградени во телевизорите, да бидат во форма на електронски весници или во некоја
друга форма за која денес сè уште не знаеме. Заедничко за сите овие компјутери е
тоа што се очекува тие да бидат постојано поврзани со што корисниците би
добивале сервиси кои се достапни постојано и од секоја локација. 

За да се овозможи развој на квалитетни апликации за новите компјутерски
платформи потребни се софтверски платформи за развој на апликации. За денешните
паметни телефони постојат повеќе такви платформи и сите тие постојано се
унапредуваат и развиваат со цел да привлечат што е можно поголем број на
развивачи на апликации. Од нивните корисници развивачите пак се очекува да се
изроди апликацијата со која луѓето ќе се приврзат и ќе ја вклопат во
секојдневниот живот. Сегментот од овој поврзан синџир на кој треба да се посвети
внимание се концептите, дизајнот и архитектурата и најдобрите практики кои треба
развивачите да ги применуваат во процесот на развој. Тие имаат една цел, да го
олеснат процесот на развој на апликации, а притоа овие апликации да се
вклопуваат со идејата и карактеристиките на новата платформа. Меѓутоа, како и
апликациите развивани во ерата во која еден сервер кој опслужува повеќе
корисници преку поделба на ресурсите во временски домен, па до засебните
апликации за персонални компјутери и до денешните популарни апликации за веб
платформата, апликациите наменети за мобилната платформа се одликуваат со многу
карактеристики со кои се издвојуваат како посебни и доста специфични за развој.
Во овие апликации од особено значење е интеракцијата и врската на корисникот со
апликацијата. Корисникот на мобилната апликација поседува многу особини кои
произлегуваат од мобилноста и од други генерални ограничувања на луѓето кога се
во движење или кога се наоѓаат во некои други невообичаени ситуации. Овие
корисници постојано ја менуваат локацијата, го менуваат опкружувањето, а со тоа
се менуваат и можностите за поврзување на нивните уреди, со еден збор го
менуваат контекстот во кој ја користат апликацијата. Проблемот како да се
идентификува овој контекст, што тој конкретно претставува и како да се искористи
во развој на апликација за новата компјутерска платформа од аспект на мобилните
уреди може да биде од огромно значење.

Во овој магистерски труд  беше презентирана архитектурата и дизајнот на мобилна
апликација со знаење за контекстот. Апликација која го препознава и користи
контекстот на корисникот претставен преку неговата локација, времето, околината
на извршување и неговите социјални врски за да се постигне подобро и по
квалитетно корисничко искуство. Беше дефинирано што претставува контекстот од
аспект на софтверски апликации, како се користи и кое е неговото значење. Беа
опишани и мобилните социјални мрежи како дел од оваа нова компјутерска
платформа, но и многу значаен извор на информации за контекстот на корисникот
поврзани со неговите социјални врски и начинот на интеграција со нив. Конечно со
овој заклучок сакаме да прикажеме како менувањето на компјутерите и
технологијата воопшто, повлекува со себе многу промени во софтверските
платформи, софтверските архитектури и начинот на развој на софтвер.

\section{Идна работа}

Преку презентираната предложена архитектура и дизајн на мобилна апликација со
знаење за контекстот опишани се детално чекорите во развојот на вакви апликации.
Идентификувањето на проблемите и предлагање на нивни решенија во овој труд не
воведува во еден нов концепт на размислување при развојот на апликации за една
сосема различна платформа. Она што недостасува, а може да се смета како
потенцијална идна работа се подобрување и тестирање на техниките за собирање и
препознавање на посложен контекст во облик на посложени ситуации и кориснички
сценарија. Препознавање и идентификување на одредени ситуации во кои може да се
наоѓа корисникот, кои се опишани со посложени контекстни информации од локација
и време е проблем кој може да се решава со по софистицирани техники за
препознавање. Дополнителен проблем претставува како ова знаење подоцна може да
се искористи за подобрување на крајниот сервис кој го имплементира апликацијата.

Интересни идеи за идна работа произлегуваат и од фактот дека ова е една нова
парадигма во компјутерските пресметки и една прилично нова платформа во која има
голем простор за развој на нови архитектури и прилагодување на некои постоечки.
Исто така скалбилноста може да претставува значаен проблем, затоа што тргнувајќи
од самиот факт дека бројот на компјутери по човек се зголемува и дека
достапноста на сервисите се очекува да биде во секое време и на секоја локација,
справувањето со зголемениот број на корисници може да претставува огромен
предизвик.
